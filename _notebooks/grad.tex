\documentclass[]{article}
\usepackage{lmodern}
\usepackage{amssymb,amsmath}
\usepackage{ifxetex,ifluatex}
\usepackage{fixltx2e} % provides \textsubscript
\ifnum 0\ifxetex 1\fi\ifluatex 1\fi=0 % if pdftex
  \usepackage[T1]{fontenc}
  \usepackage[utf8]{inputenc}
\else % if luatex or xelatex
  \ifxetex
    \usepackage{mathspec}
  \else
    \usepackage{fontspec}
  \fi
  \defaultfontfeatures{Ligatures=TeX,Scale=MatchLowercase}
\fi
% use upquote if available, for straight quotes in verbatim environments
\IfFileExists{upquote.sty}{\usepackage{upquote}}{}
% use microtype if available
\IfFileExists{microtype.sty}{%
\usepackage{microtype}
\UseMicrotypeSet[protrusion]{basicmath} % disable protrusion for tt fonts
}{}
\usepackage{hyperref}
\hypersetup{unicode=true,
            pdftitle={The Dynamics of Gradient Descent in Linear Least Squares Regression},
            pdfauthor={Akiva Lipshitz},
            pdfborder={0 0 0},
            breaklinks=true}
\urlstyle{same}  % don't use monospace font for urls
\usepackage{graphicx,grffile}
\makeatletter
\def\maxwidth{\ifdim\Gin@nat@width>\linewidth\linewidth\else\Gin@nat@width\fi}
\def\maxheight{\ifdim\Gin@nat@height>\textheight\textheight\else\Gin@nat@height\fi}
\makeatother
% Scale images if necessary, so that they will not overflow the page
% margins by default, and it is still possible to overwrite the defaults
% using explicit options in \includegraphics[width, height, ...]{}
\setkeys{Gin}{width=\maxwidth,height=\maxheight,keepaspectratio}
\IfFileExists{parskip.sty}{%
\usepackage{parskip}
}{% else
\setlength{\parindent}{0pt}
\setlength{\parskip}{6pt plus 2pt minus 1pt}
}
\setlength{\emergencystretch}{3em}  % prevent overfull lines
\providecommand{\tightlist}{%
  \setlength{\itemsep}{0pt}\setlength{\parskip}{0pt}}
\setcounter{secnumdepth}{0}
% Redefines (sub)paragraphs to behave more like sections
\ifx\paragraph\undefined\else
\let\oldparagraph\paragraph
\renewcommand{\paragraph}[1]{\oldparagraph{#1}\mbox{}}
\fi
\ifx\subparagraph\undefined\else
\let\oldsubparagraph\subparagraph
\renewcommand{\subparagraph}[1]{\oldsubparagraph{#1}\mbox{}}
\fi

\title{The Dynamics of Gradient Descent in Linear Least Squares Regression}
\author{Akiva Lipshitz}
\date{June 22, 2017}

\begin{document}
\maketitle

Arguably, one of the most powerful developments in early modern applied
mathematics is that of gradient descent, which is a technique for
solving programs of the form

\[
\arg\max _w f(w)
\]

by solving a dynamical system

\[
\mathbf{w}_{t+1} := \mathbf{w}+\alpha\lambda(\mathbf{w}\mid \mathbf{x}, \mathbf{y})
\]

where \(\mathbf{w}\) represents parameters and \(\alpha\) is a learning rate.

In this paper, we show that in gradient descent optimization for regression tasks the dynamical mode (logistic, exponential, or oscillatory, etc.) of the system defined by equation (2)
depends on which subset of the real line the learning rate of the system \(\alpha\) is an element of.
The boundaries of these subsets are significant as they represent bifurcation points in the dynamics of gradient descent.
That such results may be theoretically derived shows that after learning
rules have been written down, additional analysis must be done to
understand the asymptotic behavior and stability dynamics of the
dynamical system defined by the learning rules so as to appropriately
choose hyperparameters.

In a broad sense, this work sheds light on the space of learning processes which will hopefully inspire new developments in the study of the actual learning process of intelligent learning systems.

\subsection{Bounds on Learning Rate \(\alpha\) for which Learning Converges}
Suppose we have already derived the learning rules for a D dimensional
regression from the normality assumption. Also, we have removed all
constants of proportionality in the learning equations for the sake of
simplicity, which doesn't change the asymptotic behavior of learning.

Let \(\alpha\) be a learning rate, \(\mathbf{x}\) be a \(T\) by \(D\)
matrix, \(\mathbf{y}\) a T by 1 matrix, and \(\mathbf{w}\) a D
dimensional row vector.

\begin{align}
\Delta w_{i,t} &= -\alpha \sum\limits^N_j (y_i-\hat{y}_j)x_j \\
&= -\alpha \sum\limits^N_j (y_i-w_jx_{ij})x_{ij}\\
&= -\alpha \mathbf{x}_i\cdot\mathbf{y}\frac{1}{N} + w_{i,t}\alpha \mathbf{x}_i\cdot\mathbf{x}_i\frac{1}{N}
\end{align}

Observe in this linear task the dynamics of each weight \(w_i\) is
independent of that of any other weight. We can simplify equation (2) by
writing \(\beta_ {i1}= - \alpha\mathbf{x_i}\cdot\mathbf{y}\frac{1}{N}\)
and \(\beta_{i2}=\alpha\mathbf{x}_i \cdot\mathbf{x}_i\frac{1}{N}\), such
that

\[
\Delta w_{i,t} = \beta_{i1}+\beta_{i2}w_{it}
\]

We would like to analyze the asymptotic behavior of this dynamical
system and to do so we need an analytical expression for \(w_{i,t}\).
First, we will produce an update rule

\[
w_{i,t+1} = \beta_{i1}+w_{it} (\beta_{i2}+1)
\]

which we recognize as a one dimensional autoregressive process with an
affine term. We can recursively compose equation (5) with itself, using
\(w_{i,0} \sim \mathcal{N}(\mu, \sigma)\) as initial conditions. This is
a bit of a tedious computation that results in a closed form polynomial
expression. We simplify the indices in the computation by assuming it
holds for all \(w_i\). Thus subscripts in the computation on \(w\) refer
to iterations, with dimension implied.

\begin{align}
w_{0} &= w_0\\
w_{1} &= \beta_1+w_0(\beta_2+1)  \\
w_2  &= \beta_1+\beta_1(\beta_2+1)+w_0(\beta_2+1)^2\\
w_3 &= \beta_1 + \beta_1(\beta_2+1)+\beta_1(\beta_2+1)^2+w_0(\beta_2+1)^3\\
w_n &=w_0(\beta_2 + 1)^n +  \beta_1\sum^{n-1}_{j=0}(\beta_2+1)^j
\end{align}

This leads to somewhat of a closed form expression:

\[
w_{it} = (\beta_{i2}+1)^t w_{i0}+\beta_{i1}\sum\limits^{t-1}_{j=1}(\beta_{i2}+1)^j
\]

We are interested in the limit \(t\to \infty\) as it relates to
\(\alpha\).

\[
\lim\limits_{t\to\infty} \left [w_{it} = (\beta_{i2}+1)^t w_{i0}+\beta_{i1}\sum\limits^{t-1}_{j=1}(\beta_{i2}+1)^j \right] =\ ?
\]

Both terms in (8) converge if \(-1 < \beta_{i2}+1 < 1\). Recalling from
before \(\beta_{i2} =\alpha\mathbf{x_i}\cdot\mathbf{x_i}\frac{1}{N}\),

\begin{align}
-1 < &\beta_{i2}+1  <1 \\
-1 < &\alpha\frac{\|\mathbf{x}\|^2}{N} + 1 < 1 \\
\end{align}

There are two cases to consider and we now go through them:

If

\[
0 \le \alpha\frac{\|\mathbf{x}\|^2}{N} + 1 < 1
\]

then

\[
-1 \le \alpha\frac{\|\mathbf{x}\|^2}{N} < 0
\]

Thus

\[
-1 \le  \frac{\alpha}{N}|\mathbf{x}|^2 < 0
\]

This leads to the bounds

\[
-\frac{N}{\|\mathbf{x}\|^2} \le \alpha< 0
\]

The second case to consider is that of
\(-1 < \alpha\frac{\mathbf{x_i}\cdot\mathbf{x_i}}{N} + 1 \le 0\). Here,
using a similar thought process

\[
-2\frac{N}{\|\mathbf{x}\|^2} < \alpha \le -\frac{N}{\|\mathbf{x}\|^2}
\]

We now take the union of the sets defined by (17) and (18) as valid
\(\alpha\) values, naming it \(A\). \(A\) is expressed in terms of its
components because the inner bound is actually significant as it is the
\emph{optimal} \(\alpha\) value that leads to convergence in one step.

\[
A_i = \left( -2\frac{N}{\|\mathbf{x}_i\|^2},  -\frac{N}{\|\mathbf{x}\|^2}\right] \cup \left[-\frac{N}{\|\mathbf{x}_i\|^2}, 0 \right)
\]

\subsection{A Closed Form Expression}\label{a-closed-form-expression}

Equation (12) could have been recognized as a geometric series and is
now rewritten as such:

\[
w_{it} = (\beta_{i2}+1)^t w_0+ \beta_{i1}\frac{1-(\beta_{i2}+1)^{t}}{\beta_{i2}}
\]

Substitute \(\beta\) values and with some algebra we arrive at the
promised closed form expression. That such an equation exists is a
rarity. As such, the author believes equation (24) ought to be handled
with utmost care and placed deep in a Gringots vault for safekeeping,
far from the prying eyes of those nasty adversarial networks.

\[
w_{it} = \left[\alpha\frac{\|\mathbf{x}_i\|^2}{N} + 1\right]^t w_0-\left[\frac{\mathbf{x_i}\cdot\mathbf{y}}{\|\mathbf{x}_i\|^2} \right]\left[1-\left(\alpha\frac{\|\mathbf{x}_i\|^2}{N} + 1\right)^t\right]
\]

It is now clear to see if \(\alpha \not\in A\) then (24) diverges.

We have tested these results in python simulations and have found that
indeed with \(\alpha\) values above the upper bound
\(\alpha \le -\frac{N}{\mathbf{x}_i\cdot\mathbf{x_i}}\) , the system
converges, and the opposite for
\(\alpha > -\frac{N}{\| \mathbf{x}_i\|^2}\).

\begin{figure}[htbp]
\centering
\includegraphics{../images/Asymptotic\%20Convergence\%20of\%20Gradient\%20Descent\%20for\%20Linear\%20Regression\%20Least\%20Squares\%20Optimization_files/Asymptotic\%20Convergence\%20of\%20Gradient\%20Descent\%20for\%20Linear\%20Regression\%20Least\%20Squares\%20Optimization_12_1.png}
\caption{Asymptotic Convergence of Gradient Descent for Linear
Regression Least Squares Optimization\_12\_1}
\end{figure}

\subsection{The Dynamics of the Learning
Process}\label{the-dynamics-of-the-learning-process}

Having obtained a nice analytical expression for valid \(\alpha\)
values, we would like to understand the actual learning dynamics.How is
asymptotic convergence affected by the choice of \(\alpha\)? What is the
value of the limit in equation (14)?

There are a few interesting initial observations to make.

(\textbf{1}) From equation (4), we can easily see that if
\[\hat{y}_j=y_j\] then \(\Delta w_{i,t}=0\). Thus the true solution is a
stable point regardless of \(\alpha\).

(\textbf{2}) Equation (20) is either monotonically increasing or
decreasing. In the limit \(t \to \pm\infty\), all lower order terms drop
out and the rate of convergence is of the order
\(\mathcal{O}(\alpha^t)\).

(\textbf{3}) We can then write the characteristic timescale of
convergence \(\tau = \frac{1}{\alpha^t}\) which is exponentially small.
Thus we will observe very fast convergence.

It is worthwhile as an exercise to study the dynamics of the learning
system under the extremal values of \(\alpha\).

\subsubsection{\(\alpha> 0\),\emph{unstable}}

From the definition of \(A\), if \(\alpha > 0\) the system diverges
exponentially.

\begin{figure}[htbp]
\centering
\includegraphics{../images/Asymptotic\%20Convergence\%20of\%20Gradient\%20Descent\%20for\%20Linear\%20Regression\%20Least\%20Squares\%20Optimization_files/Asymptotic\%20Convergence\%20of\%20Gradient\%20Descent\%20for\%20Linear\%20Regression\%20Least\%20Squares\%20Optimization_9_1.png}
\caption{Asymptotic Convergence of Gradient Descent for Linear
Regression Least Squares Optimization\_9\_1}
\end{figure}

\subsubsection{\(\alpha = 0\),\emph{stable}}
In this case, the weights should diverge linearly. However, because
\(\beta_{i,1}\) depends on \(\alpha\) and \(\beta_{i1}\) is also the
constant multiple in the geometric series, the sum itself vanishes and
the trajectory is stationary.

\begin{figure}[htbp]
\centering
\includegraphics{../images/Asymptotic\%20Convergence\%20of\%20Gradient\%20Descent\%20for\%20Linear\%20Regression\%20Least\%20Squares\%20Optimization_files/Asymptotic\%20Convergence\%20of\%20Gradient\%20Descent\%20for\%20Linear\%20Regression\%20Least\%20Squares\%20Optimization_9_2.png}
\caption{png}
\end{figure}

\subsubsection{\(-\frac{N}{\|\mathbf{x}_i\|} < \alpha < 0\), \emph{stable}}

\begin{figure}[htbp]
\centering
\includegraphics{../images/Asymptotic\%20Convergence\%20of\%20Gradient\%20Descent\%20for\%20Linear\%20Regression\%20Least\%20Squares\%20Optimization_files/Asymptotic\%20Convergence\%20of\%20Gradient\%20Descent\%20for\%20Linear\%20Regression\%20Least\%20Squares\%20Optimization_9_3.png}
\caption{Asymptotic Convergence of Gradient Descent for Linear
Regression Least Squares Optimization\_9\_3}
\end{figure}

\subsubsection{\(\alpha = -\frac{N}{\|\mathbf{x}_i\|^2}\),\emph{stable}}
Plugging this into (24) yields an expression \[
w_t = 0^t\left(w_0 + \frac{\mathbf{x}_i\cdot\mathbf{y}}{\|\mathbf{x}_i\|^2}\right)+\frac{\mathbf{x}_i\cdot\mathbf{y}}{\|\mathbf{x}_i\|^2}
\] This is actually interesting because the system converges in one
iteration. The first term vanishes for \(t>0\), such that the closed
form solution is
\(\frac{\mathbf{x}_i\cdot\mathbf{y}}{\|\mathbf{x}_i\|^2}\)

\begin{figure}[htbp]
\centering
\includegraphics{../images/Asymptotic\%20Convergence\%20of\%20Gradient\%20Descent\%20for\%20Linear\%20Regression\%20Least\%20Squares\%20Optimization_files/Asymptotic\%20Convergence\%20of\%20Gradient\%20Descent\%20for\%20Linear\%20Regression\%20Least\%20Squares\%20Optimization_9_4.png}
\caption{png}
\end{figure}

\subsubsection{\(\lim\limits_{k \to -2^+} \alpha = -k\frac{N}{\|\mathbf{x}_i\|}\),\textbf{stable}}
Recall from the definiton of \(A\) that its left bound is open. As such,
the dynamics of learning are convergent for values of \(\alpha\)
infinitesimally close to \(2\).

\begin{figure}[htbp]
\centering
\includegraphics{../images/Asymptotic\%20Convergence\%20of\%20Gradient\%20Descent\%20for\%20Linear\%20Regression\%20Least\%20Squares\%20Optimization_files/Asymptotic\%20Convergence\%20of\%20Gradient\%20Descent\%20for\%20Linear\%20Regression\%20Least\%20Squares\%20Optimization_9_7.png}
\caption{png}
\end{figure}

\subsubsection{\(\alpha = -2\frac{N}{\|\mathbf{x}_i\|^2}\), \emph{unstable}}
By plugging in this value of \(\alpha\), we get an non-converging
oscillator. \[
w_t =(-1)^t w_0 + \frac{\mathbf{x}_i \cdot \mathbf{y}}{\|\mathbf{x}_i\|^2}\left((-1)^t -1\right)
\]

By neglecting the terms with constant magnitude, we can rewrite (26) to
emphasize its nature as a divergent oscillator. If we look back at our
work, (26) oscillates only because gradient descent looks for the
direction of descent, which is the negative of the error gradient with
respect to weights.

\begin{figure}[htbp]
\centering
\includegraphics{../images/Asymptotic\%20Convergence\%20of\%20Gradient\%20Descent\%20for\%20Linear\%20Regression\%20Least\%20Squares\%20Optimization_files/Asymptotic\%20Convergence\%20of\%20Gradient\%20Descent\%20for\%20Linear\%20Regression\%20Least\%20Squares\%20Optimization_9_8.png}
\caption{png}
\end{figure}

\subsection{Dynamical Bifurcations}\label{dynamical-bifurcations}

At this point we have identified some significant \(\alpha\) values and
studied the dynamics of the system under such values. To review, we
observed stationary dynamics for \(\alpha=0\), logistic growth or decay
for \(\alpha = -N/{\| x \|^2}\), and oscillatory divergence for
\(\alpha = -2N/{\|x\|^2}\). Now, notice our equations are continuous for
all values of \(\alpha\). Thus, with equation (24) we can continuously
interpolate between these the distinct dynamical regimes.

The boundary points of the set \(A\) are dynamical bifurcation points.
Suppose the boundary points of \(A\) are used to dissect the real line
into disjoint subsets. The qualitative behavior of learning dynamics is
distinct for values of \(\alpha\) picked from each of these subsets.

\subsection{Conclusion}\label{conclusion}

We have derived exact analytical bounds on \(\alpha\) values which lead
to learning convergence. Although linear regression has a closed form
solution, that such results exist is exciting. Secondly, auxilary
calculations reveal exponentially small convergence timescales. It shows
that understanding the learning behavior of gradient descent dynamical
systems is actually quite a tractable problem. This ought to inspire
efforts to understand the learning process of more complex optimization
tasks. This is practically useful as with deeper understanding comes
more powerful algorithms. In the longer run, it will be extremely
valuable to the effort to decipher the fundamental algorithms underlying
intelligent, learning, systems.

\end{document}
